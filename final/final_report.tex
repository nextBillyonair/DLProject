%------------------------------------------------------------------------------
%   PACKAGES AND OTHER DOCUMENT CONFIGURATIONS
%------------------------------------------------------------------------------

\documentclass[twoside,twocolumn]{article}

%\usepackage[sc]{mathpazo} % Use the Palatino font
%\usepackage[T1]{fontenc} % Use 8-bit encoding that has 256 glyphs
%\linespread{1.05} % Line spacing - Palatino needs more space between lines
%\usepackage{microtype} % Slightly tweak font spacing for aesthetics

\usepackage[english]{babel} % Language hyphenation and typographical rules

% Document margins
\usepackage[hmarginratio=1:1,top=32mm,left=20mm,right=20mm,columnsep=20pt]{geometry}
% Custom captions under/above floats in tables or figures
\usepackage[hang, small,labelfont=bf,up,textfont=it,up]{caption}
\usepackage{booktabs} % Horizontal rules in tables

\usepackage{enumitem} % Customized lists
\setlist[itemize]{noitemsep} % Make itemize lists more compact
\usepackage{textcomp}

% Allows abstract customization
\usepackage{abstract}
% Set the "Abstract" text to bold
\renewcommand{\abstractnamefont}{\normalfont\bfseries}
% Set the abstract itself to small italic text
\renewcommand{\abstracttextfont}{\normalfont\small\itshape}

\usepackage{fancyhdr} % Headers and footers
\pagestyle{fancy} % All pages have headers and footers
\fancyhead{} % Blank out the default header
\fancyfoot{} % Blank out the default footer
\fancyhead[C]{\thetitle}
\fancyfoot[RO,LE]{\thepage} % Custom footer text

\usepackage{titling} % Customizing the title section

\usepackage{hyperref} % For hyperlinks in the PDF
\usepackage{amsmath}
\usepackage{amssymb}

\usepackage{tikz}
\usetikzlibrary{bayesnet, arrows, positioning, fit, arrows.meta, shapes}

\usepackage{color}
\usepackage{caption}
\usepackage{subcaption}

\usepackage{graphicx}


\captionsetup[figure]{labelfont={bf},textfont=normalfont}

%------------------------------------------------------------------------------
%   TITLE SECTION
%------------------------------------------------------------------------------

\setlength{\droptitle}{-4\baselineskip} % Move the title up

\pretitle{\begin{center}\Huge\bfseries} % Article title formatting
\posttitle{\end{center}} % Article title closing formatting

\title{DL Project Final Report:\\
       Shakespeare - English Sequence to Sequence Modeling}
\author{%
\textsc{Morris Kraicer} \\[1ex]
\normalsize Johns Hopkins University \\
\normalsize \href{mailto:mkraice1@jhu.edu}{mkraice1@jhu.edu}
 \and
 \textsc{Riley Scott} \\[1ex]
\normalsize Johns Hopkins University \\
\normalsize \href{mailto:rscott39@jhu.edu}{rscott39@jhu.edu}
 \and
  \textsc{William Watson} \\[1ex]
\normalsize Johns Hopkins University \\
\normalsize \href{mailto:billwatson@jhu.edu}{billwatson@jhu.edu}
}

\date{}%\today} % Leave empty to omit a date
% \renewcommand{\maketitlehookd}{%

% }

%------------------------------------------------------------------------------

\begin{document}

% Print the title
\maketitle

%------------------------------------------------------------------------------
%   ARTICLE CONTENTS
%------------------------------------------------------------------------------

%------------------------------------------------

\begin{abstract}
% NN and NMT + attention + LSTM
\noindent
We present our attempt to create an English-Shakespeare Sequence to Sequence
Model, and its application to create a pseudo supervised style transfer system
for text. We experimented with different Encoder-Decoder Architectures with and
without Attention Mechanisms. We discuss our data procurement and processing
to help facilitate learning, and present our experimental results.
\end{abstract}

%------------------------------------------------
\section{Introduction}
We applied a suite of Encoder-Decoder models with varying attention
mechanisms (and lack of attention mechanisms) along with teacher forcing rates
to a corpus of Modern English - Original Shakespeare data. Discussion of our
data is in Section \ref{sec:data},
our processing methods are described in Section \ref{sec:preprocess}.
We used our neural machine translation systems to
apply a supervised style transfer between the two forms of writing.
More specifically, we tested models currently
used in decoding languages (Section \ref{sec:models}) and how they can
be used in this capacity. We provide results for several model
combinations and discuss the successes and failures of each system
(Section \ref{sec:results}). We also describe important implementation details
in Section \ref{sec:implementation}. Our final results are described in
Section \ref{sec:style-transfer} for our style transfer system.
Diagrams for our models can be found in the appendix.
%------------------------------------------------
\section{Data}
\label{sec:data}
\subsection{Description}
\subsection{Inital Problems}
%------------------------------------------------
\section{Preprocessing}
\label{sec:preprocess}
\subsection{Proper Nouns}
\subsection{NLTK Tokenization}
\subsection{EOS/SOS Tokens}
\subsection{Data Split}
%------------------------------------------------
\section{Model Architecture}
\label{sec:models}
\subsection{Overview}
\subsubsection{RNN}
\begin{equation}
  \label{eq:rnn}
  h_t = \tanh(W_{ih} x_t + b_{ih}  +  W_{hh} h_{t-1} + b_{hh})
\end{equation}
\subsubsection{GRU}
\begin{equation}
  \label{eq:gru}
  \begin{split}\begin{array}{ll}
    r_t = \sigma(W_{ir} x_t + b_{ir} + W_{hr} h_{t-1} + b_{hr}) \\
    z_t = \sigma(W_{iz} x_t + b_{iz} + W_{hz} h_{t-1} + b_{hz}) \\
    n_t = \tanh(W_{in} x_t + b_{in} + r_t \circ (W_{hn} h_{t-1}+ b_{hn})) \\
    h_t = (1 - z_t) \circ n_t + z_t \circ h_{t-1} \\
  \end{array}\end{split}
\end{equation}
\subsubsection{Bidirectional GRU}
\begin{equation}
  \label{eq:bidirectional}
  \begin{split}
    \begin{array}{ll}
      \overrightarrow{h_f} = \operatorname{GRU}(\overrightarrow{input})\\
      \\
      \overleftarrow{h_b} = \operatorname{GRU}(\overleftarrow{input})\\
      \\
      h_o = \overrightarrow{h_f} \,\,\Vert \,\, \overrightarrow{h_b}\\
    \end{array}
  \end{split}
\end{equation}
\subsection{Encoders}
\subsection{Decoders}
\subsection{Attention}
\subsection{Teacher-Forcing}
%------------------------------------------------
\section{Implementation}
\label{sec:implementation}
\subsection{Batching}
\subsection{GPU/CPU}
\subsection{Metrics}
\subsubsection{Masked NLL Loss}
\subsubsection{BLEU Scores}
We use the Bilingual Evaluation Understudy (BLEU) score as one of our criteria
for the quality of model translations. Outputs are in the range [0, 1].
A BLEU score of 1 indicates a perfect translation to the reference texts;
this means we need not obtain a perfect
score in order to have a good translation model.

BLEU scores are calculated for individual sentences and averaged across the
whole corpus. Intelligibility and grammatical correctness are not taken into
account when calculating BLEU scores. According to the original paper by
Papineni et al. \cite{papineni2002bleu}, we compute BLEU as follows:

\begin{equation}
  \operatorname{BLEU} = \operatorname{BP} \cdot \exp \left( \sum_{n=1}^N w_n \log p_n \right)
\end{equation}

\noindent
where $p_n$ is the geometric average of the modified $n$-gram precisions
and $\operatorname{BP}$ is the brevity penalty, explained below.
(\cite{papineni2002bleu} uses $N=4$, uniform weights $w_n = \frac{1}{N}$.)

The brevity penalty $\operatorname{BP}$ is used for translations that are
too short compared to the reference. We define BP as:

\begin{equation}
  \operatorname{BP} = \begin{cases}
    1 & \text{if } c > r \\
    e^{(1-r/c)} & \text{if } c \leq r
  \end{cases}
\end{equation}
\noindent
with effective reference corpus length $r$ and candidate translation length $c$.

%------------------------------------------------
\section{Results: Shakespeare-English}
\label{sec:results}
%------------------------------------------------
\section{Style-Transfer:\\English-Shakespeare}
\label{sec:style-transfer}
%------------------------------------------------
\section{Conclusion \& Future Work}
\label{sec:conclusion}
%------------------------------------------------







% OLD STUFF, SLOWLY MOVE INTO TOP IF NEEDED, CHANGE TENSE

\section{Data Procurement}
% \label{sec:data}
Most of our data was procured from \cite{xu2012paraphrasing}.
This includes all of Shakespeares plays translated line by line into modern
English. However, since the data was aligned, not all of the original lines
from the plays are included (the sentences could not be aligned properly).
\footnote{\url{https://github.com/cocoxu/Shakespeare}}
\begin{figure}[ht]
    \centering
    \begin{tabular}{ |l|r| }
        \hline
        \textbf{File}
          & \textbf{Line Count}\\
        \hline
        train.snt.aligned & 18,444 \\ \hline
        dev.snt.aligned & 2,107 \\ \hline
        test.snt.aligned & 528 \\ \hline
        total & 21,079 \\ \hline
    \end{tabular}

    \caption{Data Pair Counts for Shakespeare--English Corpus (Sparknotes)}
    \label{fig:data-lines-sparknotes}
\end{figure}

\begin{figure}[ht]
    \centering
    \begin{tabular}{ |l|r| }
        \hline
        \textbf{File}
          & \textbf{Line Count}\\
        \hline
        train.snt.aligned & 9,069 \\ \hline
        dev.snt.aligned & 1,036 \\ \hline
        test.snt.aligned & 260 \\ \hline
        total & 10,365 \\ \hline
    \end{tabular}

    \caption{Data Pair Counts for Shakespeare--English Corpus (Enotes)}
    \label{fig:data-lines-enotes}
\end{figure}

The training vocabulary sizes are 11,538 source (original) words and
9,024 target (modern) words for the Sparknotes set. For enotes, there are
9004 source words, 7497 target words.
%------------------------------------------------

\section{Preprocessing}
% \label{sec:preprocess}
We describe our data processing algorithms to create test, development,
and training samples in a simple and easy format.
\subsection{SOS and EOS}
% talk about use
We encapsulate every sentence with two special tokens: SOS and EOS.
The Start of Sentence token (SOS) signals the start of a sequence, and allows
us to map our first real word to one that most likely starts a sentence.

We use the End of Sentence (EOS) token to signal the end of the sequence, and
always comes after punctuation.

By incorporating these special tokens, we can signal to the model the start and
end of a sequence, and help training overall. This is done during the run, and
is not present in our data files (inserted at run time).

\subsection{Proper Nouns}
Proper nouns are unique in both corpus, and have direct translations. In order
to reduce vocabulary size and aggregate the learning of all proper nouns,
we replace all proper nouns with the following token: propn. Thus our model
should learn to map propn to propn, and can utilize the encoding to learn the
most likely token following its usage. We use
SpaCy's\footnote{\url{https://spacy.io}} nlp models to identify
proper nouns in each sentence, and replace the tokens.

\subsection{Miscellaneous}
% lower, nltk tokenizing, spliting punct etc
Additionally, we lower case all input. We use the Natural Language ToolKit's
(NLTK) tokenization algorithm
to split punctation, contrations, etc for each word.

\subsection{Train, Validation, Test Split}
% look at code, provide stats (Vocab size, # sent, etc.)
We randomly shuffle and split the combined preprocessed dataset into three sets:
Train, Dev, and Test. We opted for a 87.5/10/2.5 split to reduce the appearance
of unkown tokens (UNK). We provide statistics for the data.
% TODO: Provide Data Stats

\begin{figure*}[ht]
    \centering
    \resizebox{\textwidth}{!}{%
    \begin{tabular}{ |c|l|l| }
        \hline
            $\#$
            & \textbf{Original Sentence}
            & \textbf{Modern Sentence}\\
        \hline
        1
            & there ’ s beggary in the love that can be reckoned .
            & it would be a pretty stingy love if it \\
         &
         & could be counted and calculated . \\
        \hline
        2
            & poor souls , they perished .
            & the poor people died . \\
        \hline
        3
            & the propn ’ s abused by some most villainous knave ,
            & the propn is being tricked by some crook ,  \\
         & some base notorious knave , some scurvy fellow .
         & some terrible villain , some rotten bastard . \\
        \hline
        4
            & my best way is to creep under his gaberdine .
            & the best thing to do is crawl under his cloak . \\
        \hline
        5
            & when they do choose , they have the wisdom by their wit to lose .
            & when they choose , they only know how to lose . \\
        \hline
        6
            & her blush is guiltiness , not modesty .
            & she blushes from guilt , not modesty . \\
        \hline
        7
            & go , hang yourselves all !
            & go hang yourselves , all of you ! \\
        \hline
        8
            & then , if ever thou propn acknowledge it ,
            & then , if you dare to acknowledge it , \\
         & i will make it my quarrel .
         & i ’ ll take up my quarrel with you . \\
        \hline
        9
            & and lovers ' absent hours more tedious
            & and lovers ' hours are a hundred and  \\
         & than the dial eightscore times !
         & sixty times longer than normal ones ! \\
        \hline
        10
            & i have no great devotion to the deed and
            & i do n ’ t really want to do this , \\
         & yet he hath given me satisfying reasons .
         & but he ’ s given me good reasons . \\
        \hline
        \end{tabular}}
    \caption{Sample Original-Modern Sentence Pairs, Processed
      (Without SOS/EOS Tokens)}
    \label{fig:sample-pairs}
\end{figure*}

%------------------------------------------------

\section{Architectures}
% \label{sec:architecture}
% go more detailed? or can they not handle the math lol
% these descriptions are good starters, but we can be mroe detailed.
We seek to develop several models to improve our translations, incorporating
context, attention, and novel training methods. It will
incorporate an encoder-decoder style model \cite{cho2014learning}
\cite{sutskever2014sequence}.

% IDEA on subsectioning:
\subsection{Encoders}
The encoder (or \emph{inference network}) receives an input token sequence
$\vec{x} = \left[{x_1,\hdots, x_n}\right]$ of length $n$ and processes
this to create an output encoding. The result is a sequence
$\vec{h} = \left[{h_1, \cdots, h_{T_x}}\right]$ that
maps to the input sequence $\vec{x}$.
\subsubsection{Baseline RNN Encoder}
For the baseline encoder, we implemented a simple Embedding + RNN encoder.
This accepts an input sequence $\vec{x}$ and encoders the sequence using the lookup
embeddings and forward context.
\begin{equation}
  \label{eq:rnn}
  h_t = \tanh(W_{ih} x_t + b_{ih}  +  W_{hh} h_{t-1} + b_{hh})
\end{equation}
Howver, this is the simplest model, prone to the vanishing gradient problem
on large sequences. In addition, this model has the lowest capacity to learn.
Nonetheless, we will present results for our baseline.

\begin{figure*}[ht]
    \centering
    \begin{tabular}{ |c|c|c|c|c| }
        \hline
        $\#$
          & \textbf{Encoder}
          & \textbf{Decoder}
          & \textbf{Attention}
          & \textbf{Teacher Forcing (Percent)} \\
        \hline
        1 & RNN & RNN & None & None \\ \hline
        2 & GRU & GRU & None & None \\ \hline
        3 & Bidirectional GRU & GRU & None & None \\ \hline
        4 & Bidirectional GRU & GRU & Concat & None \\ \hline
        5 & Bidirectional GRU & GRU & General & None \\ \hline
        6 & Bidirectional GRU & GRU & Concat & 0.5 \\ \hline
        7 & Bidirectional GRU & GRU & General & 0.5 \\ \hline
        8 & Bidirectional GRU & GRU & Concat & 1.0 \\ \hline
        9 & Bidirectional GRU & GRU & General & 1.0 \\ \hline
    \end{tabular}

    \caption{Planned Model Experiments (Subject to change depending on results)}
    \label{fig:model-experiments}
\end{figure*}

\subsubsection{GRU Encoder}
An obvious improvement to our encoding scheme would be to replace the RNN layer
with a GRU. A GRU encodes a forward sequence using more complex equations to
improve capacity and learning.
\begin{equation}
  \label{eq:gru}
  \begin{split}\begin{array}{ll}
    r_t = \sigma(W_{ir} x_t + b_{ir} + W_{hr} h_{t-1} + b_{hr}) \\
    z_t = \sigma(W_{iz} x_t + b_{iz} + W_{hz} h_{t-1} + b_{hz}) \\
    n_t = \tanh(W_{in} x_t + b_{in} + r_t \circ (W_{hn} h_{t-1}+ b_{hn})) \\
    h_t = (1 - z_t) \circ n_t + z_t \circ h_{t-1} \\
  \end{array}\end{split}
\end{equation}
GRUs help with vanishing gradients, increases our models capacity to learn,
and uses less parameters than the LSTM layer. Hence, for our purposes, we
used a GRU over the LSTM.

\subsubsection{Bidirectional GRU Encoder}
An extension to our encoding scheme will consider the full context of words
immediately before an after it. This is done by running a GRU layer on the
forward and backward sequence and combining each tensor. Hence we use a
bidirectional GRU as laid forth in \cite{bahdanau2014neural}.
\begin{equation}
  \label{eq:bidirectional}
  \begin{split}
    \begin{array}{ll}
      \overrightarrow{h_f} = \operatorname{GRU}(\overrightarrow{input})\\
      \\
      \overleftarrow{h_b} = \operatorname{GRU}(\overleftarrow{input})\\
      \\
      h_o = \overrightarrow{h_f} \,\,\Vert \,\, \overrightarrow{h_b}\\
    \end{array}
  \end{split}
\end{equation}
PyTorch concatenates the resulting tensors, doubling the hidden output size
of a normal GRU. Other libraries allow for concatenatation, averaging,
and summing. We use PyTorch's implementation of a bidirectional GRU.

\subsubsection{Implementation}
All encoders share the same model architecture, except for the recurrent layer.
We use PyTorch's implementation for the RNN, GRU, and Bidirectional GRU.
We also add embedding layers. PyTorch's recurrent layers naturally support
multiple layers and dropout, and can be set through CLI args
\texttt{--num-layers} and \texttt{--lstm-dropout}. The hidden size is set by
\texttt{--hidden-size}. The defaults are 1, 0.1, and 256, respectively.


\subsection{Decoders}
We will describe the decoder algorithm, and experiment with two different
recurrent layers: RNN and GRU. We cannot use a Bidirectional GRU because
we do not know the full decoded sequence (and hence why we are decoding).

Decoders take in the last translated token, starting with an SOS token on a
new batch. It applies an embedding layer, followed by an optional dropout.

We then have two options:
\begin{enumerate}
  \item Attention (General, Concat, etc.)
  \item No Attention
\end{enumerate}
In the case of attention, we apply one of the attention schemes (described in
the next section) to the encoder output, given our current decoding
hidden state. We concatenate the attention results with our input embedding.

Without attention, we just use the input embedding and ignore the encoder
outputs.

We apply a recurrent layer to the attended tensor, using the hidden state
provided. After applying a linear layer and log softmax, we output our result
for evaluation.

\subsection{Attention Mechanisms}
% \label{sec:attention}
Attention mechanisms have been shown to improve sequence to sequence
translations from Bahdanau et al \cite{bahdanau2014neural}, and further work
from Luong et al \cite{luong2015effective} examines global vs local approached
to attention-based encoder-decoders. Common attention mechanisms are:
% we will probably use dot & concat attention
\begin{equation}
    score(h_t, h_s) =
    \begin{cases}
        v_a^T \cdot \tanh (W_a [h_t \| h_s]) & \text{\emph{concat}} \\
        h_t^T \cdot W_a \cdot h_s & \text{\emph{general}} \\
        h_t^T \cdot h_s & \text{\emph{dot}} \\
    \end{cases}
\end{equation}
where $h_t$ is the current target hidden state, and $h_s$ is the encoder output.
To compute scores, which are used to create attention weights, we apply a
softmax:
\begin{equation}
  a_t(s) = \frac{\exp(score(h_t, h_s))}{\sum_{s'}\exp(score(h_t, h_{s'}))}
\end{equation}
Using these scores, we create a context vector, which is just the batch matrix
multiplication between our attention weights and the encoder outputs.
We will focus on general attention and concat attention in our
experiments.

\subsection{Teacher Forcing}
In terms of training, an encoder-decoder system can either accept the target
token or the model's prediction as input during the decoding step. When we use
the target token, this is known as teacher forcing, and is shown to be favored
during initial training iterations, but should be backed off to use the
model's own predicitons, as it will exhibit instability in the translations
otherwise \cite{lamb2016professor}.

We hope to build in a system to decay the teacher forcing percentage over time,
instead of our current implementation that checks a random number againist the
hyperparameter. However, we can bypass this effectively by reloading a model
and changing the teacher forcing parameter provided.

%------------------------------------------------
\section{Planned Experiments}
We plan to experiment with several model permutations, as outlined in Figure
~\ref{fig:model-experiments}. We do not test every permutation since that would
take too much time, and have selected several runs that would provide us enough
results to determine the best model for this problem.

\begin{figure*}[ht]
    \centering
    \begin{tabular}{ |c|c|c|c| }
        \hline
        $\#$
          & \textbf{Task}
          & \textbf{Status}
          & \textbf{Team Member}\\
        \hline
        1 & Data Procurement & DONE & All \\ \hline
        2 & Preprocessing & DONE & Riley, Bill \\ \hline
        3 & RNN Encoder & DONE & Riley \\ \hline
        4 & GRU Encoder & DONE & Morris \\ \hline
        5 & Bidirectional GRU & DONE & Morris \\ \hline
        6 & RNN Decoder & TESTING & Morris \\ \hline
        7 & GRU Decoder & TESTING & Bill \\ \hline
        8 & Attention Models & TESTING & Bill \\ \hline
        9 & Teacher Forcing & DONE & Bill \\ \hline
        10 & Vocabulary Building & DONE & Bill \\ \hline
        11 & Train/Eval Support Code & DONE & Bill \\ \hline
        12 & GPU Support & DONE & Riley \\ \hline
        13 & Batching & DONE & Bill \\ \hline
        14 & Experiments & IN PROGRESS & Morris, Riley \\ \hline
    \end{tabular}

    \caption{Current Progress}
    \label{fig:current-status}
\end{figure*}

%------------------------------------------------

\section{Roadblocks and Problems}
We have hit two major roadblocks since our proposal, and discuss our apporaches
to mediating the issues.

\subsection{Data Learnability}
In our first tests, we used data procured from \cite{xu2012paraphrasing}.
There were two parallel datasets, one from Sparknotes, the other from enotes. We
originally planned to use the Sparknotes version, as the data set was larger
(21079 sentence pairs). However, when training, the models had difficulty
accounting for the reordering Sparknotes used in their translations. Hence, we
decieded to switch to enotes, which is a smaller (10365 sentence pairs), but
more aligned corpus. We theorize that data that has less reordering will converge
faster to a desired result. Since the dataset is smaller, we use less batches,
and training time trivially improves. We hope for decent results by switching.


\subsection{GPU/CPU Training Times}
Systems like these take a long time to learn the data, and one of our major
roadblocks is training time. For a normal CPU run, 2 epochs takes 11-30 minutes,
depending on computer specs. This would put our experiments total running time
at over 2-3 days per experiment (18-27 days total). This does not give us much
leeway to verify our
apporach. Hence, we describe in Section ~\ref{sec:gpu} our intial resutls
in GPU compatability and speedups. These timings were done on the Sparknotes
dataset.

%------------------------------------------------

\section{Additional Implementation\\Details}
\subsection{Batching}
In order to faciliate faster training, and less noisy gradients, we felt it
imperative to introduce batching of sentences. We batched similiar sentences
according to source sentence length (encoder input). This allows us to reduce
the number of batches to loop through and take advantage of torch operations.

\subsection{GPU Compatability}
% \label{sec:gpu}
Initial training is slow on a cpu, with a Bidirectional GRU Encoder
+ GRU Decoder + Concat Attention estimated at 11-30 minutes per 2 epochs,
varying on batch size (32 and 128 tested). In order to improve training time,
we have allowed an optional parameter to use a gpu. Inital run on batch size 128
yielded a training speed of 111 seconds per 2 epochs. Hence our experiments will
take only 4-6 days, and can be distributed across multiple GPUs for even less
time.

Our group has an estimated 3 GPUs (Morris:1; Riley:2). This will help us
distribute our experiments across several computers and allow us time to adjust
experiments based upon our initial findings. If need be, we will us Google
credits to fund more experiments.

%------------------------------------------------

\section{Current Status and\\Expectation}
%Where are we at now + what are we going to do.
See Figure ~\ref{fig:current-status} for our current standing on this project.
Our notation is as follows:
\begin{enumerate}
  \item DONE - Fully implemented, tested, no further work needed
  \item IN PROGRESS - Currently being implemented or worked on. Some inital results
    but more work is required for full functionality.
  \item TESTING - Fully functional, but currently being tested for bugs, etc.
\end{enumerate}

While most of the coding is done, and the last few model architectures are
being tested, we estimate the longest task to be completed is model
experimentation, and may take the next two-three weeks to complete.

We will automate all planned experiments, examine the results, and use the best
model as our style-transfer. We will then begin our final training session
on both directions and provide results. We hopefully expect training of our
experiments completed by the end of Thanksgiving break, and can pick a model
to perform style-transfer for English-Shakespeare examples.

%------------------------------------------------



%------------------------------------------------

% References
\bibliographystyle{abbrv}
\bibliography{final_report}

%------------------------------------------------

\clearpage
\appendix
\onecolumn
\section{Model Architecture Diagrams}
\begin{figure}[ht]
  \centering
  \begin{tikzpicture}

    % Define nodes
    \node[rectangle] (encoder) {Encoder};
    \node[rectangle, below=2cm of encoder] (decoder) {Decoder};
    \node[rectangle, above=0.75cm of encoder, xshift=-1cm] (s) {$source$};
    \node[rectangle, above=0.75cm of encoder, xshift=1cm] (hidden1) {$None$};
    \node[rectangle, below=0.75cm of encoder, xshift=-1cm] (out1) {$encodings$};
    \node[rectangle, below=0.75cm of encoder, xshift=1cm] (hidden2) {$hidden$};
    \node[rectangle, left=0.75cm of decoder] (t) {$token$};
    \node[rectangle, below=0.75cm of decoder, xshift=-1cm] (out2) {$new$ $token$};
    % \node[rectangle, below=0.75cm of decoder, xshift=1cm] (hidden3) {$hidden$};

    % Connect the nodes
    \edge {s, hidden1} {encoder};
    \edge {encoder} {out1, hidden2};
    \edge {out1, t} {decoder};
    \edge {decoder} {out2};
    \edge {out2} {t};
    \path [->] (decoder) edge [bend right=15] node {} (hidden2);
    \path [->] (hidden2) edge [bend right=15] node {} (decoder);

  \end{tikzpicture}
  \caption{Model Architecture Overview for Encoder-Decoder.}
  \label{fig:encoder-decoder}
\end{figure}
\begin{figure}[ht]
  \centering
  \begin{tikzpicture}

    % Define nodes
    \node[rectangle] (rnn) {RNN/GRU};
    \node[rectangle, below=0.75cm of rnn, xshift=-1cm] (out) {$output$};
    \node[rectangle, below=0.75cm of rnn, xshift=1cm] (hidden2) {$hidden$};
    \node[rectangle, above=0.75cm of rnn, xshift=-1cm] (embed) {Embed};
    \node[rectangle, above=0.75cm of embed] (input) {$source$};
    \node[rectangle, right=0.75cm of input] (hidden1) {$hidden$};

    % Connect the nodes
    \edge {input} {embed};
    \edge {embed, hidden1} {rnn};
    \edge {rnn} {out, hidden2};

  \end{tikzpicture}
  \caption{Model Architecture for Encoder.}
  \label{fig:encoder}
\end{figure}
\begin{figure}[ht]
  \centering
  \begin{tikzpicture}

    % Define nodes
    \node[rectangle] (rnn) {RNN/GRU};
    \node[rectangle, above=0.75cm of rnn] (embed) {Embed};
    \node[rectangle, above=0.75 of embed] (input) {$input$};
    \node[rectangle, right=1cm of input] (hidden1) {$hidden$};
    \node[rectangle, below=0.75cm of rnn] (linear) {Linear};
    \node[rectangle, below=0.75cm of linear] (logs) {Log Softmax};
    \node[rectangle, below=0.75cm of logs] (out) {$output$};
    \node[rectangle, right=1.25cm of out] (hidden2) {$hidden$};

    % Connect the nodes
    \edge {input} {embed};
    \edge {embed, hidden1} {rnn};
    \edge {rnn} {linear, hidden2};
    \edge {linear} {logs};
    \edge {logs} {out};

  \end{tikzpicture}
  \caption{Decoder with No Attention.}
  \label{fig:decoder-no-attn}
\end{figure}
\begin{figure}[ht]
  \centering
  \begin{tikzpicture}

    % Define nodes
    \node[rectangle] (rnn) {RNN/GRU};
    \node[rectangle, above=0.75 of rnn, xshift=-1cm] (cat) {Concat};
    \node[rectangle, above=0.75cm of cat] (attn) {Attn};
    \node[rectangle, left=0.75cm of attn] (embed) {Embed};
    \node[rectangle, above=0.75 of embed] (input) {$input$};
    \node[rectangle, above=0.75cm of attn] (encoder) {$encoding$};
    \node[rectangle, right=1cm of encoder] (hidden1) {$hidden$};
    \node[rectangle, below=0.75cm of rnn] (linear) {Linear};
    \node[rectangle, below=0.75cm of linear] (logs) {Log Softmax};
    \node[rectangle, below=0.75cm of logs] (out) {$output$};
    \node[rectangle, right=1cm of out] (hidden2) {$hidden$};

    % Connect the nodes
    \edge {input} {embed};
    \edge {encoder, hidden1} {attn};
    \edge {attn, embed} {cat};
    \edge {cat, hidden1} {rnn};
    \edge {rnn} {linear, hidden2};
    \edge {linear} {logs};
    \edge {logs} {out};

  \end{tikzpicture}
  \caption{Decoder with Attention.}
  \label{fig:decoder-attn}
\end{figure}
\begin{figure}[ht]
  \centering
  \begin{tikzpicture}

    % Define nodes
    \node[rectangle] (out) {$output$};
    \node[rectangle, above=0.75cm of out] (bmm) {BMM};
    \node[rectangle, above=0.75cm of bmm, xshift=-1cm] (soft) {Softmax};
    \node[rectangle, above=0.75 of soft] (lin2) {Linear};
    \node[rectangle, above=0.75 of lin2] (tanh) {$\tanh$};
    \node[rectangle, above=0.75 of tanh] (lin1) {Linear};
    \node[rectangle, above=0.75 of lin1] (cat) {Concat};
    \node[rectangle, above=0.75 of cat, xshift=-1cm] (hidden) {$hidden$};
    \node[rectangle, above=0.75 of cat, xshift=1cm] (encodings) {$encodings$};

    % Connect the nodes
    \edge {encodings, hidden} {cat};
    \edge {cat} {lin1};
    \edge {lin1} {tanh};
    \edge {tanh} {lin2};
    \edge {lin2} {soft};
    \edge {soft, encodings} {bmm};
    \edge {bmm} {out};

  \end{tikzpicture}
  \caption{Concat (Bahdanau) Attention Layer.}
  \label{fig:concat-attn}
\end{figure}
\begin{figure}[ht]
  \centering
  \begin{tikzpicture}

    % Define nodes
    \node[rectangle] (out) {$output$};
    \node[rectangle, above=0.75cm of out] (bmm) {BMM};
    \node[rectangle, above=0.75cm of bmm, xshift=-1cm] (soft) {Softmax};
    \node[rectangle, above=0.75 of soft] (bilinear) {Bilinear};
    \node[rectangle, above=0.75 of bilinear, xshift=-1cm] (hidden) {$hidden$};
    \node[rectangle, above=0.75 of bilinear, xshift=1cm] (encodings) {$encodings$};

    % Connect the nodes
    \edge {encodings, hidden} {bilinear};
    \edge {lin2} {soft};
    \edge {soft, encodings} {bmm};
    \edge {bmm} {out};

  \end{tikzpicture}
  \caption{General Attention Layer.}
  \label{fig:general-attn}
\end{figure}

\end{document}
