%------------------------------------------------------------------------------
%   PACKAGES AND OTHER DOCUMENT CONFIGURATIONS
%------------------------------------------------------------------------------

\documentclass[twoside,twocolumn]{article}

%\usepackage[sc]{mathpazo} % Use the Palatino font
%\usepackage[T1]{fontenc} % Use 8-bit encoding that has 256 glyphs
%\linespread{1.05} % Line spacing - Palatino needs more space between lines
%\usepackage{microtype} % Slightly tweak font spacing for aesthetics

\usepackage[english]{babel} % Language hyphenation and typographical rules

% Document margins
\usepackage[hmarginratio=1:1,top=32mm,left=20mm,right=20mm,columnsep=20pt]{geometry}
% Custom captions under/above floats in tables or figures
\usepackage[hang, small,labelfont=bf,up,textfont=it,up]{caption}
\usepackage{booktabs} % Horizontal rules in tables

\usepackage{enumitem} % Customized lists
\setlist[itemize]{noitemsep} % Make itemize lists more compact
\usepackage{textcomp}

% Allows abstract customization
\usepackage{abstract}
% Set the "Abstract" text to bold
\renewcommand{\abstractnamefont}{\normalfont\bfseries}
% Set the abstract itself to small italic text
\renewcommand{\abstracttextfont}{\normalfont\small\itshape}

\usepackage{fancyhdr} % Headers and footers
\pagestyle{fancy} % All pages have headers and footers
\fancyhead{} % Blank out the default header
\fancyfoot{} % Blank out the default footer
\fancyhead[C]{\thetitle}
\fancyfoot[RO,LE]{\thepage} % Custom footer text

\usepackage{titling} % Customizing the title section

\usepackage{hyperref} % For hyperlinks in the PDF
\usepackage{amsmath}
\usepackage{amssymb}

\usepackage{tikz}
\usetikzlibrary{bayesnet, arrows, positioning, fit, arrows.meta, shapes}

\usepackage{color}
\usepackage{caption}
\usepackage{subcaption}

\usepackage{graphicx}


\captionsetup[figure]{labelfont={bf},textfont=normalfont}

%------------------------------------------------------------------------------
%   TITLE SECTION
%------------------------------------------------------------------------------

\setlength{\droptitle}{-4\baselineskip} % Move the title up

\pretitle{\begin{center}\Huge\bfseries} % Article title formatting
\posttitle{\end{center}} % Article title closing formatting

\title{Deep Learning Project Proposal:\\Shakespeare - English Sequence to Sequence Modeling}
\author{%
\textsc{Morris Kraicer} \\[1ex]
\normalsize Johns Hopkins University \\
\normalsize \href{mailto:mkraice1@jhu.edu}{mkraice1@jhu.edu}
 \and
 \textsc{Riley Scott} \\[1ex]
\normalsize Johns Hopkins University \\
\normalsize \href{mailto:rscott39@jhu.edu}{rscott39@jhu.edu}
 \and
  \textsc{William Watson} \\[1ex]
\normalsize Johns Hopkins University \\
\normalsize \href{mailto:billwatson@jhu.edu}{billwatson@jhu.edu}
}

\date{}%\today} % Leave empty to omit a date
% \renewcommand{\maketitlehookd}{%

% }

%------------------------------------------------------------------------------

\begin{document}

% Print the title
\maketitle

%------------------------------------------------------------------------------
%   ARTICLE CONTENTS
%------------------------------------------------------------------------------

% \section{Introduction}

% TODO

% Abstract

% Inspiration

% Data
% 	web scrap
% 	label via aligner

% Preprocessing
% 	Tokenize!!!!!
% 		NUM
% 		PROPER NOUNS
% 		PUNCTUATION
% 		USE SPACY FOR POS TAGS

% Models
% 	Baseline
% 		Seq2Seq Baseline
% 			No attention, no Bidirectional
% 			encoder RNN-> RNN
% 		GRU->GRU
% 		bidirectional GRU -> Bidirectional GRU
% 		With Attentions
% 		Teacher Forcing

% 	Use Aligner to measure quality of data, and learnability.
% 		Not learnable -> still train, but when report, do backwards, suggest data is difficult
% 		Learnable -> do aligner do well, then models

% Final Tricks:
% 	Num -> Num
% 	PN -> PN
% 	Punct -> Punct

% Expectation:
% 	Hopefully provides good translations
% 	Hopefully can do a pseudo style transfer on English
% 	Funny translations? 50 Shades of Grey

%------------------------------------------------

\begin{abstract}
% NN and NMT + attention + LSTM
\noindent
We present a sequence-to-sequence neural translation model with attention. More specifically, we describe the different encoder-decoder models used; our implementation of a Bidirectional Gated Recurrent Unit (GRU) encoder; and an Encoder and Decoder with attention. We will incrementally build up our models to examine training performance on Shakespeare-English aligned data, and attempt to perform a pseudo style-transfer on text.
\end{abstract}

\section{Introduction}
For our final project we would like to develop several models to translate modern English to Shakespearean English and also Shakespearean English to modern English. Our data will be a dataset of shakespeares plays aligned with their modern English translations. We will train several different architectures and compare how well they each perform this task. A unique challenge poised in the problem is Shakespeare's usage of meter and word order, and it will be interesting to see if this influences our results.

\section{Data Procurement}
Most of our data will come from the cited github \cite{xu2012paraphrasing}. This includes all of Shakespeares plays translated line by line into modern English . If needed, we may also use Sonnets, which we may need to align ourselves through either the Berkley or Giza aligner. \footnote{\url{https://github.com/cocoxu/Shakespeare}}

\section{Preprocessing}
We will use spacy.io to tokenize our dataset to replace proper nouns with pronouns.\footnote{\url{https://spacy.io}} In addition, we will tokenize numbers and punctuation to reduce our vocabulary size. Finally, we will lowercase the data. We will randomly sample a 70/20/10 split for train/dev/test on the full merged dataset.

\section{Architectures}
We seek to develop several models to improve our translations, incorporating context, attention, and novel training methods.

\subsection{Baseline RNN Sequence to Sequence}
Our baseline model will be a simple RNN sequence to sequence model. It will accept a source sentence $s$ and will decode it to an output sequence $t$. A simple RNN will only incorporate previous states, i.e. source words, during prediction, and might be prone to the vanishing gradient problem. It will incorporate an encoder-decoder style model \cite{cho2014learning} \cite{sutskever2014sequence}.
\subsection{GRU Sequence to Sequence}
We will improve our baseline through the use of the GRU layer, to hopefully offset any vanishing gradients. In addition, we use a GRU over an LSTM to help bound the number of parameters used in the model for training purposes, as the GRU and LSTM fix issues regarding long term dependencies within sequences.
\subsection{Bidirectional Model}
In order to consider the context of words before and after, we will alter the model to use a bidirectional GRU \cite{bahdanau2014neural}, and hopefully see better translations.
\subsection{Attention Mechanisms}
Attention mechanisms have been shown to improve sequence to sequence translations from Bahdanau et al \cite{bahdanau2014neural}, and further work from Luong et al \cite{luong2015effective} examines global vs local approached to attention-based encoder-decoders.
\subsection{Teacher Forcing}
In terms of training, an encoder-decoder system can either accept the target token or the model's prediction as input during the decoding step. When we use the target token, this is known as teacher forcing, and is shown to be favored during initial trainging iterations, but should be backed off to use the model's own predicitons, as it will exhibit instability in the translations otherwise \cite{lamb2016professor}.

\section{Roles and Responsibilities}
We intend to split the project into the following subsections according to Figure \ref{fig:roles}.

\begin{figure}
  \centering
  \begin{tabular}{ |l|r| }
      \hline
      \textbf{Roles} & \textbf{Members} \\
      \hline
      Data Procurement & Morris, Riley, Bill \\ \hline
      Preprocessing Scripts & Riley, Bill \\ \hline
      Baseline Model & Riley \\ \hline
      GRU Extension & Morris \\ \hline
      Bidirectional Extension & Morris \\ \hline
      Attention Mechanisms & Bill \\ \hline
      Train and support code & Bill \\ \hline
      Training experiments & Morris, Riley \\ \hline
      Writeup and analysis & Morris, Riley, Bill \\ \hline
  \end{tabular}
  \caption{Role Distribution}
  \label{fig:roles}
\end{figure}

\section{Expectations}
We hope to provide a model that can except English source sentences and decode them into Shakespearean English, in a pseudo style-transfer for text using ideas found in neural machine translation models. We hope to apply this model on modern literature and evaluate the quality of the system as a whole.

% References
\bibliographystyle{abbrv}
\bibliography{proposal}

\end{document}
