%------------------------------------------------------------------------------
%   PACKAGES AND OTHER DOCUMENT CONFIGURATIONS
%------------------------------------------------------------------------------

\documentclass[twoside,twocolumn]{article}

%\usepackage[sc]{mathpazo} % Use the Palatino font
%\usepackage[T1]{fontenc} % Use 8-bit encoding that has 256 glyphs
%\linespread{1.05} % Line spacing - Palatino needs more space between lines
%\usepackage{microtype} % Slightly tweak font spacing for aesthetics

\usepackage[english]{babel} % Language hyphenation and typographical rules

% Document margins
\usepackage[hmarginratio=1:1,top=32mm,left=20mm,right=20mm,columnsep=20pt]{geometry}
% Custom captions under/above floats in tables or figures
\usepackage[hang, small,labelfont=bf,up,textfont=it,up]{caption}
\usepackage{booktabs} % Horizontal rules in tables

\usepackage{enumitem} % Customized lists
\setlist[itemize]{noitemsep} % Make itemize lists more compact
\usepackage{textcomp}

% Allows abstract customization
\usepackage{abstract}
% Set the "Abstract" text to bold
\renewcommand{\abstractnamefont}{\normalfont\bfseries}
% Set the abstract itself to small italic text
\renewcommand{\abstracttextfont}{\normalfont\small\itshape}

\usepackage{fancyhdr} % Headers and footers
\pagestyle{fancy} % All pages have headers and footers
\fancyhead{} % Blank out the default header
\fancyfoot{} % Blank out the default footer
\fancyhead[C]{\thetitle}
\fancyfoot[RO,LE]{\thepage} % Custom footer text

\usepackage{titling} % Customizing the title section

\usepackage{hyperref} % For hyperlinks in the PDF
\usepackage{amsmath}
\usepackage{amssymb}

\usepackage{tikz}
\usetikzlibrary{bayesnet, arrows, positioning, fit, arrows.meta, shapes}

\usepackage{color}
\usepackage{caption}
\usepackage{subcaption}

\usepackage{graphicx}


\captionsetup[figure]{labelfont={bf},textfont=normalfont}

%------------------------------------------------------------------------------
%   TITLE SECTION
%------------------------------------------------------------------------------

\setlength{\droptitle}{-4\baselineskip} % Move the title up

\pretitle{\begin{center}\Huge\bfseries} % Article title formatting
\posttitle{\end{center}} % Article title closing formatting

\title{DL Project Proposal:\\Shakespeare to English}
\author{%
\textsc{Morris Kraicer} \\[1ex]
\normalsize Johns Hopkins University \\
\normalsize \href{mailto:mkraice1@jhu.edu}{mkraice1@jhu.edu}
 \and
 \textsc{Riley Scott} \\[1ex]
\normalsize Johns Hopkins University \\
\normalsize \href{mailto:rscott39@jhu.edu}{rscott39@jhu.edu}
 \and
  \textsc{William Watson} \\[1ex]
\normalsize Johns Hopkins University \\
\normalsize \href{mailto:billwatson@jhu.edu}{billwatson@jhu.edu}
}

\date{}%\today} % Leave empty to omit a date
% \renewcommand{\maketitlehookd}{%

% }

%------------------------------------------------------------------------------
\DeclareMathOperator*{\argmax}{arg\,max}
\newcommand{\qdist}[1]{\ifmmode\langle#1\rangle\else\textlangle#1\textrangle\fi}
\renewcommand{\vec}[1]{\mathbf{#1}}

\begin{document}

% Print the title
\maketitle

%------------------------------------------------------------------------------
%   ARTICLE CONTENTS
%------------------------------------------------------------------------------

% \section{Introduction}

% TODO

%------------------------------------------------

\begin{abstract}
% NN and NMT + attention + LSTM
\noindent
%We present a sequence-to-sequence neural translation model with attention. More specifically, we describe the network layers used; our implementation of a Bidirectional Long Short-Term Memory (LSTM) layer; and an Encoder and Decoder with attention. We additionally present visualizations of the attention layer as well as a discussion on training results from the given data.
Add abstract to be fancy
\end{abstract}

\section{Introduction}
\cite{bahdanau2014neural}
\cite{luong2015effective}
\cite{liu2016neural}
\cite{cho2014learning}
\cite{sutskever2014sequence}
\cite{papineni2002bleu}
Abstract

Inspiration

Data
	web scrap
	label via aligner

Preprocessing
	Tokenize!!!!!
		NUM
		PROPER NOUNS
		PUNCTUATION
		USE SPACY FOR POS TAGS

Models
	Baseline
		Seq2Seq Baseline
			No attention, no Bidirectional
			encoder RNN-> RNN
		GRU->GRU
		bidirectional GRU -> Bidirectional GRU
		With Attentions
		Teacher Forcing

	Use Aligner to measure quality of data, and learnability.
		Not learnable -> still train, but when report, do backwards, suggest data is difficult
		Learnable -> do aligner do well, then models

Final Tricks:
	Num -> Num
	PN -> PN
	Punct -> Punct

Expectation:
	Hopefully provides good translations
	Hopefully can do a pseudo style transfer on English
	Funny translations? 50 Shades of Grey
\section{Data Procurement}

\section{Preprocessing}

\section{Archectectures}

\subsection{Baseline RNN Sequence to Sequence}

\subsection{GRU Sequence to Sequence}

\subsection{Bidirectional Model}

\subsection{Attention Mechanisms}

\subsection{Teacher Forcing}

\section{Roles and Responsibilities}

\section{Expectations}

% References
\bibliographystyle{abbrv}
\bibliography{proposal}

\end{document}
